% make rubber recompile for biber:
% rubber: watch ausarbeitung.bcf 
% rubber: onchange ausarbeitung.bcf 'biber ausarbeitung'

\documentclass{amsart}

% packages
\usepackage[utf8]{inputenc}
\usepackage{mathtools}
\usepackage[ngerman]{babel}
%\usepackage{paralist}
\usepackage{extarrows}
\usepackage{dsfont}
\usepackage{amsthm}

% bibliography
\usepackage[backend=biber]{biblatex}
\DeclareFieldFormat{postnote}{#1}
\addbibresource{bibliography.bib}

% page layout
\usepackage{geometry}
\geometry{verbose,a4paper,tmargin=3.5cm,bmargin=2.5cm,lmargin=2.6cm,rmargin=2.6cm}


% theorem styles
\theoremstyle{plain}
\newtheorem{proposition}[subsection]{Satz}
\newtheorem{corollary}[subsection]{Korollar}
\newtheorem{lemma}[subsection]{Lemma}

\theoremstyle{definition}
\newtheorem{definition}[subsection]{Definition}
\newtheorem{remark}[subsection]{Bemerkung}
\newtheorem*{notation}{Notation}


% custom environments
%\newenvironment{enumeratei}
%	{\begin{compactenum}[(i)]}
%	{\end{compactenum}}
%

% custom commands
\newcommand{\projspace}{\mathds{P}}
\newcommand{\pic}{\operatorname{Pic}}
\newcommand{\divgrp}{\operatorname{Div}}
\newcommand{\rightarrowiso}{\xlongrightarrow{\sim}}
\newcommand{\affinespace}{\mathds{A}}
\renewcommand{\hom}{\operatorname{Hom}}

\begin{document}


%\section{Varietäten als Schemata}
%\label{section-varietäten-als-schemata}
%
%

\section{Jakobische Varietät}
\label{section-jakobische-varietaet}

Eine Besonderheit elliptischer Kurven ist die Tatsache, dass diese Kurven ihre eigene Jakobische sind.
Die Jakobische einer Kurve ist eine Varietät, deren $k$-wertigen Punkte eine zur $Pic^0$-Gruppe der Kurve isomorphe Gruppenstruktur tragen.
Der Beweis der Existenz der Jakobischen für allgemeine Kurven ist nichttrivial, im Falle von elliptischen Kurven (die über die im letzten Vortrag vorgestellte Gruppenstruktur der $k$-wertigen Punkte ihre eigene Jakobische ist) aber im Rahmen unserer Möglichkeiten aber machbar.
In diesem Abschnitt wird (ohne Beweise) ein kurzer Einblick in die Theorie der Jakobischen gegeben.

\begin{definition}
	\label{def-gruppenobjekt}
	Sei $C$ eine Kategorie, in der endliche Produkte existieren. Insbesondere besitzt $C$ ein terminales Objekt $*$.
	Ein Objekt $G \in C$ zusammen mit Morphismen
	\begin{align*}
		& \operatorname{mul} \colon E \times E \rightarrow E \\
		& \operatorname{inv} \colon E \rightarrow E \\
		& \operatorname{eins} \colon * \rightarrow E
	\end{align*}
	so, dass die Diagramme
	\begin{equation*}
	\end{equation*}
	kommutieren, heißt {\it Gruppenobjekt}.
	Kommutiert auch
	\begin{equation*}
	\end{equation*}
	so heißt $G$ {\it abelsches} Gruppenobjekt.
\end{definition}

\begin{remark}
	Ein Gruppenobjekt $G$ kann über die Yoneda-Einbettung auch folgendermaßen charakterisiert werden:

	\noindent Ist $S \in C$ ein beliebiges Objekt, so wird $\hom(S, G)$ vermöge
	\begin{align*}
		& f \cdot g \coloneqq \operatorname{mul} \circ (f \times g) \\
		& 1 = \operatorname{eins} \circ (S \rightarrow *) \\
		& f^{-1} = \operatorname{inv} \circ f
	\end{align*}
	für $f, g \in \hom(S, G)$ zur Gruppe. Ein Morphismus $\phi \in \hom(S, T)$ ergibt mit dieser Definition einen Morphismus von Gruppen
	\begin{equation*}
		- \circ \phi \colon \hom(T, G) \rightarrow \hom(S, G).
	\end{equation*}
	Ist umgekehrt für alle $S \in C$ eine in $S$ funktorielle (abelsche) Gruppenstruktur auf $\hom(S, G)$ gegeben, so ergeben die natürlichen Transformationen
	\begin{align*}
		\eta_{\operatorname{mul}} \colon &
		\begin{cases}
			\hom(-, G) \times \hom(-, G) \rightarrow \hom(-, G) \\
			\hom(S, G) \times \hom(S, G) \ni (f, g) \mapsto f + g \in \hom(S, G)
		\end{cases} \\
		\eta_{\operatorname{inv}} \colon &
		\begin{cases}
			\hom(-, G) \rightarrow \hom(-, G) \\
			\hom(S, G) \ni f \mapsto f^{-1} \in \hom(S, G)
		\end{cases} \\
		\eta_{\operatorname{eins}} \colon &
		\begin{cases}
			\hom(-, *) \rightarrow \hom(-, G) \\
			\hom(S, *) \ni f \mapsto 1 \in \hom(S, G)
		\end{cases} \\
	\end{align*}
	nach dem Yoneda-Lemma ($X \mapsto \hom(-, X)$ ist voll-treue Einbettung in die Funktorkategorie $C \mapsto \operatorname{Set}$) die Morphismen $\operatorname{mul}$, $\operatorname{inv}$ und $\operatorname{eins}$ wie in \ref{def-gruppenobjekt}.
\end{remark}

\begin{definition}
	Ein Gruppenobjekt in der Kategorie der Schemata über einem Schema $S$ heißt {\it algebraische Gruppe über $S$}.
\end{definition}

\begin{definition}
	Sei $k$ ein Körper. Eine algebraische Gruppe $A$ über $k$ heißt {\it abelsche Varietät über $k$}, falls $A$ geometrische integer und eigentlich über $k$ ist.
\end{definition}

\noindent Der Name {\it abelsche} Varietät ist durch folgenden Satz gerechtfertigt:
\begin{proposition}
	Eine abelsche Varietät über $k$ ist notwendig projektiv und ein abelsches Gruppenobjekt.
\end{proposition}

\begin{proposition}
	Sei $C$ eine glatte, geometrische zusammenhängende, projektive Kurve über $k$ vom Geschlecht $g$. Dann existiert eine abelsche Varietät $J$ über $k$ der Dimension $g$ so, dass 
	\begin{equation*}
		J(L) \cong \pic^0(C_L)
	\end{equation*}
	für jede Körpererweiterung $L|k$ mit $C(L) \neq \emptyset$, funktoriell in $L$.
\end{proposition}




\section{Elliptische Kurven}
\label{section-elliptische-kurven}

In diesem Abschnitt wird die Gleichheit elliptischer Kurven und ihrer Jakobischen bewiesen.
Zu Beginn werden elliptische Kurven als nichtsinguläre Kurven vom Geschlecht 1 mit einem ausgezeichneten Basispunkt definiert.
Wir werden sehen, dass elliptische Kurven sich alternativ auch als nichtsinguläre, durch eine Weierstraß-Gleichung gegebene Kurven charakterisieren lassen. \\
Im zweiten Teil wird eine Isomorphie der $\pic^0$-Gruppe mit der durch die im letzten Vortrag kennen gelernte Addition auf einer elliptischen Kurve gegebenen Gruppenstruktur bewiesen werden.
Da die Addition auf elliptischen Kurven sich als Morphismus herausstellen wird, ist die elliptische Kurve damit ihre eigene Jakobische.
Im Folgenden wird wieder die klassische Sprache der Varietäten in Anlehnung an \cite{silverman}.

\begin{definition}
	Ein Paar $(E, O)$, wobei $E$ eine nichtsinguläre Kurve vom Geschlecht 1, $O \in E$ ein Punkt sind, heißt {\it elliptische Kurve}.
	Die elliptische Kurve heißt {\it definiert über $k$}, falls $E$ über $k$ definiert ist und $O \in E(k)$. Insbesondere besitzt $E$ einen $k$-rationalen Punkt.
\end{definition}

\begin{notation}
	Im Folgenden sei stets $k$ ein Körper mit algebraischem Abschluss $K$, $(E, O)$ eine über $k$ definierte elliptische Kurve. 
	Statt $(E, O)$ schreiben wir auch einfach $E$, womit implizit eine elliptische Kurve mit Basispunkt $O$ gemeint ist.
\end{notation}

\begin{proposition}
	\label{prop-existenz-weierstrass-koordinaten}
	Es existieren $x, y \in k(E)$ so, dass
	\begin{equation*}
		\phi = [x, y, 1] : E \rightarrow C
	\end{equation*}
	ein Basispunkt-erhaltender Isomorphismus von $E$ mit einer durch eine Weierstraß-Gleichung definierten Kurve $C \subset \projspace^2$ über $k$ ist. \\
	$C$ ist also gegeben durch
	\begin{equation*}
		Y^2 + a_1 XY + a_3 Y = X^3 + a_2 X^2 + a_4 X + a_6
	\end{equation*}
	mit $a_1, \dots, a_6 \in k$ und
	\begin{equation*}
		\phi(O) = (0 : 1 : 0).
	\end{equation*}
\end{proposition}

\begin{definition}
	$x, y \in k(E)$ wie in \ref{prop-existenz-weierstrass-koordinaten} heißen {\it Weierstraß-Koordinaten}.
\end{definition}

\begin{proposition}
	Seien $(x, y)$, $(x', y')$ zwei Paare von Weierstraß-Koordinaten. Dann existieren $u, r, s, t \in k, u \neq 0$ so, dass
	\begin{align*}
		x &= u^2 x' + r \\
		y &= u^3 y' + s u^2 x' + t.
	\end{align*}
\end{proposition}

\begin{proposition}
	Sei $C$ eine nichtsinguläre Kurve über $k$, gegeben durch eine Weierstraß-Gleichung. Dann ist $(C, (0 : 1 : 0))$ eine elliptische Kurve.
\end{proposition}

\begin{lemma}
	Seien $P, Q \in E$. Dann gilt
	\begin{equation*}
		(P) \sim (Q) \Leftrightarrow P = Q
	\end{equation*}
\end{lemma}

\begin{proposition}
	\label{prop-isomorphism-pic}
	Die Funktion
	\begin{equation*}
		\kappa \colon \begin{cases}
			E \rightarrow \pic^0(E) \\
			P \mapsto \overline{(P) - (O)}
		\end{cases}
	\end{equation*}
	ist ein Isomorphismus abelscher Gruppen.
\end{proposition}

\begin{proposition}
	Die natürliche Abbildung
	\begin{equation*}
		\divgrp_k^0 \mapsto \pic_k^0(E)
	\end{equation*}
	ist surjektiv. $\kappa$ aus \ref{prop-isomorphism-pic} beschränkt sich zu 
	\begin{equation*}
		E(k) \rightarrowiso \pic_k^0(E).
	\end{equation*}
\end{proposition}

\begin{corollary}
	Die Sequenz
	\begin{equation*}
		1 \rightarrow k(E)^{\times} \rightarrow \divgrp_k^0(E) \rightarrow \pic_k^0(E) \rightarrow 0
	\end{equation*}
	ist exakt.
\end{corollary}

\begin{corollary}
	Sei $D \in \divgrp_k(E)$. Dann gilt
	\begin{equation*}
		D \sim 0 \Leftrightarrow \deg D = 0 \text{ und } \sum_{P \in E} n_P P = 0.
	\end{equation*}
\end{corollary}

\begin{proposition}
	Die Abbildungen
	\begin{align*}
		\operatorname{add} \colon &
		\begin{cases}
			E \times E \rightarrow E \\
			(P, Q) \mapsto P + Q
		\end{cases} \\
		\operatorname{inv} \colon &
		\begin{cases}
			E \rightarrow E \\
			P \mapsto -P
		\end{cases} \\
		\operatorname{null} \colon &
		\begin{cases}
			\affinespace^0 \rightarrow E \\
			0 \mapsto O
		\end{cases}
	\end{align*}
	sind Morphismen von Varietäten. $E$ wird vermöge dieser Abbildungen zum Gruppenobjekt in der Kategorie der Varietäten über $k$.
\end{proposition}

\printbibliography

\end{document}

